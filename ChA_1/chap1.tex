\documentclass[10 pt, a4paper, draft]{article}
\usepackage{polski}
\usepackage[utf8]{inputenc}
\usepackage{listings}

\author{Bartłomiej Sacharski}
\title{WIZUT Bash Guide - Rozdział I (draft)}

\begin{document}
\maketitle
\tableofcontents
\newpage

\section{Pierwsze spotkanie z konsolą}
\subsection{cd}
%TODO write about cd

\subsection{pwd}
Większość terminali jest skonfigurowana w taki sposób aby przed znakiem zachęty \texttt{\$} wyświetlać nazwę katalogu w którym obecnie pracujemy. Jeżeli okazałoby się jednak, że terminal nie wyświetla tej informacji, lub nie jest ona dla nas wystarczająca, z pomocą przychodzi polecenie \texttt{pwd} (\textbf{p}rint name of current/\textbf{w}orking \textbf{d}irectory). Przykład zastosowania takiego polecenia mamy poniżej:
\newline
%TODO Fix tilda sign
\texttt{[beton@xps \textasciitilde]\$ pwd } \newline %TODO Change pwd in example
\texttt{/home/beton} \newline 
Jak widać na przykładzie, po zastosowaniu polecenia, dowiadujemy się że pracujemy w katalogu domowym użytkownika \textit{beton} - potwierdza to także znak \textasciitilde przed znakiem zachęty \$ (\textasciitilde jest aliasem dla nazwy katalogu domowego obecnego użytkownika).

\subsection{. i ..}
%TODO write about . and ..

\subsection{uruchamianie programów i plików wykonywalnych}
%TODO write about executing executables

\newpage
\section{Proste operacje na plikach i katalogach}
\subsection{touch}
Jednym ze sposobów na utworzenie pustego pliku z poziomu wiersza poleceń jest użycie komendy \texttt{touch}. Przykład utworzenia pliku \textit{koza} będzie wyglądał następująco:
\newline
\texttt{\$ touch \textit{koza}}
%TODO Add more info about modyfing timestamps

\subsection{mv}
Aby móc przenieść pliki i katalogi, lub zmienić nazwę danego obiektu skorzystamy z polecenia \texttt{mv} (\textbf{m}o\textbf{v}e).

\subsection{rm}

\newpage
\section{Edytory tekstu}
\subsection{pico i nano}
\subsection{Vi i Vim}

\newpage
\section{Uzyskiwanie pomocy}
\subsection{man}
\subsection{info}
\subsection{apropos}

\end{document}