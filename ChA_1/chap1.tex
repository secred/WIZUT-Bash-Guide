\documentclass[10 pt, a4paper, draft]{article}
\usepackage{polski}
\usepackage[utf8]{inputenc}
\usepackage{listings}
\usepackage{amsmath}
\usepackage{amssymb}
\usepackage{amsfonts}
\usepackage{draftwatermark}

\author{Bartłomiej Sacharski}
\title{WIZUT Bash Guide - Rozdział I}

\begin{document}
\maketitle
\tableofcontents
\newpage

\section{Pierwsze spotkanie z konsolą}
\subsection{cd}
Jedną z najczęściej wykonywanych w terminalu akcji będzie prawdopodobnie zmiana katalogu w którym obecnie pracuje użytkownik. Do tego celu służy prosta komenda \texttt{cd} (\textbf{c}hange \textbf{d}irectory). Przykład zastosowania komendy mamy poniżej:
\newline
%TODO add cd
%TODO possibly add cd switches
\subsection{pwd}
Większość terminali jest skonfigurowana w taki sposób aby przed znakiem zachęty \texttt{\$} wyświetlać nazwę katalogu w którym obecnie pracujemy. Jeżeli okazałoby się jednak, że terminal nie wyświetla tej informacji, lub nie jest ona dla nas wystarczająca, z pomocą przychodzi polecenie \texttt{pwd} (\textbf{p}rint name of current/\textbf{w}orking \textbf{d}irectory). Przykład zastosowania takiego polecenia mamy poniżej:
\newline
%TODO Fix tilda sign
\texttt{[beton@xps \textasciitilde]\$ pwd } \newline %TODO Change pwd in example
\texttt{/home/beton} \newline 
Jak widać na przykładzie, po zastosowaniu polecenia, dowiadujemy się że pracujemy w katalogu domowym użytkownika \textit{beton} - potwierdza to także znak \textasciitilde przed znakiem zachęty \$ (\textasciitilde jest aliasem dla nazwy katalogu domowego obecnego użytkownika).

\subsection{. i ..}
%TODO write about . and ..

\subsection{uruchamianie programów i plików wykonywalnych}
%TODO write about executing executables

\newpage
\section{Proste operacje na plikach i katalogach}
\subsection{touch}
Jednym ze sposobów na utworzenie pustego pliku z poziomu wiersza poleceń jest użycie komendy \texttt{touch} (jest to właściwie efekt uboczny działania tego polecenia, ale mimo to bardzo przydatny). Przykład utworzenia pliku \textit{koza} będzie wyglądał następująco:
\newline
\texttt{\$ touch \textit{koza}}
\newline
Jeżeli chcemy utworzyć na raz kilka plików możemy je utworzyć jednym poleceniem, oddzielając spacjami kolejne nazwy:
\newline
\texttt{\$ touch \textit{plik1 plik2 plik3}}
\subsubsection*{Przełączniki}
%TODO Add more info about modyfing timestamps

\subsection{mv}
Aby móc przenieść pliki i katalogi, lub zmienić nazwę danego obiektu skorzystamy z polecenia \texttt{mv} (\textbf{m}o\textbf{v}e).
%TODO write about mv
\subsubsection*{Przełączniki}
%TODO add switches

\subsection{rm}
Do skasowania plików używamy polecenia \texttt{rm}(\textbf{r}e\textbf{m}ove)
\subsubsection*{Przełączniki}
%TODO add switches

\subsection{mkdir}
Odpowiednikiem dla polecenia touch do tworzenia jest polecenie \texttt{mkdir} (\textbf{m}a\textbf{k}e \textbf{dir}ectories. Aby utworzyć przykładowy katalog \textit{koza} za pomocą wiersza poleceń skorzystamy z komendy:
\newline
\texttt{\$ mkdir \textit{koza}}
\newline 
Oczywiście, jeżeli chcielibyśmy utworzyć taki katalog wewnątrz innego katalogu, np. \textit{łąka}, jako parametr przekażemy ścieżkę do tego katalogu:
\newline
\texttt{\$ mkdir \textit{łąka/koza}}
\newline 
Jeżeli jednak próbowalibyśmy utworzyć katalog w miejscu które nie istnieje, np. łąkaa wydając polecenie:
\newline
\texttt{\$ mkdir \textit{łąkaa/koza}}
\newline
to bez zastosowania dodatkowego przełącznika \textbf{-p} konsola zwróci błąd z komunikatem:
\newline
\texttt{mkdir: nie można utworzyć katalogu `łąkaa/koza': Nie ma takiego pliku ani katalogu}
\subsubsection*{Przełączniki}
%TODO add switches

\subsection{rmdir}
Jak łatwo się domyślić - jeżeli istnieje polecenie \texttt{mkdir}, służące do tworzenia katalogów, to istnieje również polecenie \texttt{rmdir} (\textbf{r}e\textbf{m}ove empty \textbf{dir}ectories), służące do kasowania pustych katalogów. Wywołanie operacji usuwania katalogu \textit{koza} będzie identyczne jak podczas tworzenia go:
\newline
\texttt{\$ rmdir \textit{koza}}
\newline
Pozostałe operacje jak kasowanie katalogu wewnątrz innego katalogu oraz usuwanie całej hierarchii katalogów jest analogiczne do operowania poleceniem \texttt{mkdir}.
\newline
\subsubsection*{Przełączniki}
\begin{description}
\item \textbf{-p} lub \textbf{--parents} usuwa katalog wraz z jego katalogami nadrzędnymi
%TODO add more switches
\end{description}

\newpage
\section{Edytory tekstu}
\subsection{pico i nano}
\subsection{Vi i Vim}

\newpage
\section{Uzyskiwanie pomocy}
\subsection{man}
W przypadku gdy chcemy dowiedzieć się jak działa polecenie, lub sprawdzić sposób jego wywołania możemy spróbować wywołać polecenie z przełącznikiem \textbf{-?} lub \textbf{--help}. Wiele programów wyświetla wtedy bardzo ogólną informację o dostępnych przełącznikach, ich działaniu i sposobie wywołania programu. Czasem jednak potrzebujemy bardziej rozbudowanej dokumentacji - wtedy warto spróbować skorzystać z polecenia \texttt{man}(od angielskiego słowa \textbf{man}ual).
\subsubsection*{Przełączniki}
%TODO add switches
\subsection{info}
\subsection{apropos}
W przypadku gdy chcemy coś zrobić, ale nie wiemy jakiego polecenia użyć, możemy skorzystać z polecenia \texttt{apropos}. Jako argument wywołania podajemy wówczas frazę jakiej poszukujemy w instrukcji, a \texttt{apropos} przeszuka wszystkie dostępne na dysku instrukcje pod kątem tejże frazy. Przykładowo, jeżeli chcielibyśmy dowiedzieć się jakie polecenie odpowiada za tworzenie katalogu  użyjemy nowopoznanej komendy z argumentem \textit{tworzy katalog}:
\newline
%TODO add this
\end{document}