\chapter{Podstawy operowania wierszem poleceń}
Drogi Czytelniku! \newline
Witamy w pierwszym rozdziale krótkiego skryptu \doctitle.
W tej sekcji spróbujemy zapoznać Cię z podstawami obsługi wiersza poleceń systemów z rodziny Linuks. Oczywiście, rozumiemy Twoje zdziwienie - jak w czasach monitorów Full HD i interfejsów dotykowych, można korzystać jeszcze z czegoś wydaje się tak przestarzałego jak wiersz poleceń. Nic bardziej mylnego - tryb tekstowy ma się bardzo dobrze i jest używany przez miliony osób na całym świecie. Gwarantujemy, że po zapoznaniu się z całą książką sam stwierdzisz, iż niektóre operacje szybciej jest "wystukać" niż "wyklikać".

\section{Pierwsze spotkanie z konsolą}
Podczas swojej nauki będziesz korzystał z tzw. emulatora terminala\index{terminal} - w dawnych czasach było to urządzenie pozwalające na pracę z komputerem, często była to klawiatura służąca do wprowadzania informacji oraz monitor lub drukarka do wyprowadzania danych. Wraz z postępem technologicznym terminale zostały zastąpione wirtualnymi odpowiednikami - właśnie takim wirtualnym terminalem będziemy się posługiwać. Bardzo możliwe, że będziesz korzystać z PuTTY, Konsole lub gnome-terminal. Jeśli jednak Twój program nazywa się inaczej to nie przejmuj się - nie ma to w tym przypadku większego znaczenia.

Jeśli zechcesz uruchomić system z rodziny Linuks na swoim własnym komputerze możesz skorzystać z tak zwanych \textit{live cd}\index{live cd}, które pozwalają na przetestowanie systemu z poziomu płyty cd lub dvd\footnote{Spora część płyt live pozwala także na uruchamianie z poziomu pamięci podłączanej do portu USB. Jeżeli jesteś zainteresowany takim sposobem uruchamiania, to zapoznaj się z projektem \textit{UNetbootin}} bez potrzeby jego instalacji na dysku twardym. Jeżeli uznasz, że praca z systemem jest dla Ciebie przyjemna, to możesz skorzystać z opcji instalacji obok istniejącego systemu operacyjnego - podczas uruchamiania komputera, zyskasz możliwość wyboru jaki system chcesz uruchomić. Jeżeli jesteś zainteresowany przetestowaniem Linuksa na swoim komputerze poszukaj w wyszukiwarce internetowej hasła \textit{live cd linux}.

\subsection{Tuż po uruchomieniu}
Ze sporą dozą prawdopodobieństwa można założyć, że na ekranie zostanie wyświetlony tekst wyglądający mniej-więcej tak:
\begin{verbatim}
[beton@xps ~]$ 
\end{verbatim}
W tym krótkim tekście zawartych jest sporo informacji - pracujemy obecnie jako użytkownik \textit{beton} na maszynie zwanej \textit{xps} i katalogiem w którym obecnie się znajdujemy jest katalog domowy (oznaczany często znakiem \textasciitilde). Kolejnym znakiem który informuje nas o stanie terminala jest znak \$ (\index{znak zachęty}nazywany znakiem zachęty). Pojawienie się tego znaku oznacza, że terminal nie wykonuje w tym momencie żadnej operacji i można wprowadzić nowe polecenie. Co ciekawe, jeżeli zdarzy się Wam pracować na koncie z uprawnieniami administratora systemu \index{root}(konto root) zauważycie pewnie, że zamiast znaku zachęty \$ pojawi się znak \# - takie rozróżnienie jest często wykorzystywane w instrukcjach opisujących wykonanie czegoś za pomocą wiersza poleceń; polecenia wykonywane za pomocą konta administratora będą się zaczynały właśnie od tego znaku.

\subsection{Wywoływanie poleceń}
Wywoływanie wielu różnych poleceń odbywa się w dosyć szablonowy sposób:
\begin{verbatim}
komenda [-przełączniki/parametry] [argumenty]
\end{verbatim}
Po nazwie komendy możemy podać \texttt{przełączniki}, zmieniające sposób działania programu, a także \texttt{argumenty}, przekazujące do polecenia zestaw danych na których będzie ono wykonywane. Spora część programów dysponuje podwójnymi przełącznikami - jednoliterowymi, poprzedzonymi znakiem myślnika (np. \textbf{-a}) i wieloliterowymi, poprzedzonymi dwoma znakami myślnika (np. \textbf{--all}). Niektóre przełączniki można nawet łączyć ze sobą, pisząc je jeden po drugim (np. \textbf{-la}). Niestety nie wszystkie programy trzymają się tej konwencji, dlatego jeśli nie jesteś pewien poprawnego wywołania polecenia, skorzystaj z pomocy, która bardzo często jest zainstalowana na dysku twardym.

\subsection{Uzyskiwanie pomocy}
W skrypcie zamieściliśmy tylko częściowy opis poleceń - bardziej szczegółowe informacje możesz odnaleźć samodzielnie korzystając z komend opisanych poniżej.

\subsubsection{man}
W przypadku gdy chcemy dowiedzieć się jak działa polecenie, lub sprawdzić sposób jego wywołania możemy spróbować wywołać polecenie z przełącznikiem \textbf{-?} lub \textbf{--help}. Wiele programów wyświetla wtedy bardzo ogólną informację o dostępnych przełącznikach, ich działaniu i sposobie wywołania programu. Czasem jednak potrzebujemy bardziej rozbudowanej dokumentacji - wtedy warto spróbować skorzystać z polecenia \texttt{man}(\textit{ang.} \textbf{man}ual).

\subsubsection{info}
Tak jak \texttt{man}, \texttt{info} również wyświetla pomoc dotyczącą polecenia. W przeciwieństwie do poprzedniego programu, \texttt{info} jest bardziej intuicyjnym programem, który jest niestety nieco mniej popularny --- bardzo możliwe że polecenie, którego pomoc będziemy chcieli przeczytać, będzie posiadało ją dostępną poprzez czytnik \texttt{man} zamiast poprzez czytnik \texttt{info}. Mimo to warto wiedzieć o jego istnieniu, chociażby ze względu na interaktywne spisy treści, dzięki którym szybko przeniesiemy się do konkretnego miejsca w tekście.

\subsubsection{apropos}
W przypadku gdy chcemy coś zrobić, ale nie wiemy jakiego polecenia użyć, możemy skorzystać z polecenia \texttt{apropos}. Jako argument wywołania podajemy wówczas frazę jakiej poszukujemy w instrukcji, a \texttt{apropos} przeszuka wszystkie dostępne na dysku instrukcje pod kątem tejże frazy. Przykładowo, jeżeli chcielibyśmy dowiedzieć się jakie polecenie odpowiada za tworzenie katalogu użyjemy nowo poznanej komendy z argumentem \textit{tworzy katalog}:
\newline
\begin{verbatim}
beton@megas:~$ apropos tworzy katalog
catman (8)           - tworzy lub aktualizuje preformatowane strony podręcznika man
create_module (2)    - tworzy wpis ładowalnego modułu
ln (1)               - tworzy dowiązania między plikami
mandb (8)            - tworzy lub aktualizuje bufory indeksowe stron man.
mkdir (1)            - tworzy katalogi
mke2fs (8)           - tworzy linuksowy system plików ext2/3
mkfifo (1)           - tworzy FIFO (potoki nazwane)
\end{verbatim}
Jak widać polecenie przeszukało instrukcje pod kątem słów ,,tworzy'' oraz ,,katalog''. Jeżeli będziemy chcieli znaleźć całą frazę, to musimy umieścić ją w cudzysłowach np:
\newline
\begin{verbatim}
beton@megas:~$ apropos "tworzy katalog"
mkdir (1)            - tworzy katalogi
\end{verbatim}
Zawęziło to trochę listę wyników, nieprawdaż? Warto zapoznać się z dokumentacją komendy, ponieważ w skrypcie nie poruszymy jej przełączników i dodatkowych ustawień, a ich działanie może Ci kiedyś pomóc w znalezieniu tego jednego, \emph{właściwego} programu.

\subsection{Uprawnienia do plików}
\label{sec:uprawnienia}
Zanim zaczniemy przeglądać zawartość dysku twardego, powinniśmy Ci wcześniej wytłumaczyć czym są uprawnienia dostępu\index{uprawnienia dostępu} do plików i wyjaśnić dlaczego są one tak istotne podczas codziennej pracy.
\newline
\newline
W systemach z rodziny Linuks każdy plik czy też katalog, posiada domyślnie zestaw trzech wartości informujących o prawach dostępu do niego - informują one o tym, czy możesz przeglądać zawartość pliku/katalogu, zapisywać do niego zmiany czy też wykonywać plik/przeglądać katalog. Aby mechanizm działał sensownie jego twórcy zdecydowali, że należy definiować uprawnienia dla trzech przypadków:
\begin{itemize}
\item uprawnienia dla właściciela (najczęściej jest to jego twórca),
\item uprawnienia dla grupy (uprawnienia dla pewnej grupy użytkowników),
\item uprawnienia dla innych (dla użytkowników którzy nie pasują do dwóch poprzednich kategorii).
\end{itemize}
Każdy z przypadków może przyjąć pewien zestaw wartości oznaczający konkretne uprawnienia:
\begin{table}[h!t]
\begin{center}
\begin{tabular}{|c|c|c|} \hline
Zapis literowy & Zapis cyfrowy & Opis \\ \hline
\texttt{r} & 4 & Odczyt \\ \hline
\texttt{w} & 2 & Zapis \\ \hline
\texttt{x} & 1 & Wykonywanie \\ \hline
\end{tabular}
\caption{Sposóby zapisu uprawnień do plików}
\label{tab:uprawnienia}
\end{center}
\end{table}
\newline
Aby zapewnić różne uprawnienia np. odczyt i zapis sumujemy ich wartości liczbowe, co daje nam wartość 6. Oczywiście, oprócz zapisu liczbowego możemy również stosować bardziej intuicyjny zapis znakowy, np "\texttt{rw}". Chcąc sprawdzić jakie uprawnienia posiada plik możemy skorzystać z polecenia \texttt{ls} z przełącznikiem \textbf{-l} (więcej o poleceniu możesz przeczytać na stronie \pageref{sec:ls}). Abyś lepiej zrozumiał jak odczytywać uprawnienia przypisane plikowi posłużymy się plikiem \textit{fstab} znajdującym się w katalogu \textit{/etc}:
\begin{verbatim}
-rw-rw-r--.  1 root root      501 03-01 23:40 fstab
\end{verbatim}
W powyższym przykładzie właściciel pliku posiada zarówno uprawnienia odczytu jak i zapisu do pliku, tak samo jest w przypadku grupy która jest obecnie przypisana do pliku. Inni użytkownicy posiadają jedynie prawa odczytu tego pliku. Właścicielem pliku jest użytkownik \textit{root}, taka sama jest także nazwa grupy do której uprawnienia są zdefiniowane. Rozpatrzmy teraz inny przykład - utworzony własnoręcznie katalog \textit{virtual}
\begin{verbatim}
drwxr-xr-x.   2 beton root  4096 03-29 10:38 virtual
\end{verbatim}
Jak pewnie się już domyśliłeś, właściciel katalogu posiada do niego pełne uprawnienia - odczyt, zapis i możliwość przeglądania katalogu - w przypadku grupy oraz innych użytkowników możliwe jest jedynie odczytywanie i przeglądanie zawartości katalogu. Właścicielem jest tutaj użytkownik \textit{beton}, ale grupa do której uprawnienia są zdefiniowane to \textit{root}. Możliwe, że zastanawiasz się teraz co oznacza litera \texttt{d} widniejąca przed uprawnieniami - jest to oznaczanie, że dany element jest katalogiem (w przypadku pliku \textit{fstab} zamiast litery \texttt{d} stał znak \texttt{-}).
\newline
\newline
Bardzo ważne jest pamiętanie o definiowanie właściwych uprawnień pliku, jeśli przyznamy zbyt duże uprawnienia to może się okazać, że inni użytkownicy przeglądają nasze katalogi lub nawet zmieniają zawartość plików. Z drugiej strony należy też uważać aby nie ustawić zbyt restrykcyjnych uprawnień lub ich nawet nie odebrać, bo może to się skończyć błędnym działaniem systemu lub zainstalowanych programów.

\subsection{chmod}
\label{sec:chmod}
Skoro wytłumaczyliśmy Ci już czym są uprawnienia, najwyższy czas by pokazać Ci jak przyznawać, zmieniać i odbierać uprawnienia do plików i katalogów. Poleceniem które nam to umożliwi jest \texttt{chmod} (\textit{ang.} \textbf{ch}ange file \textbf{mod}es or Access Control Lists). Typowe wywołanie polecenia \texttt{chmod} zawiera \textit{przełącznik} \textit{uprawnienia} \textit{pliki} których prawa chcemy zmienić. Chcąc natychmiast zmienić uprawnienia właściciela, grupy i innych możemy posłużyć się zapisem liczbowym - przykładowo, chcąc nadać pełnie praw do pliku \textit{koza} właścicielowi, a grupie i innym odebrać wszystkie prawa wykonamy następujące polecenie:
\begin{verbatim}
[beton@xps ~]$ chmod 700 koza
\end{verbatim}
Wynik działania polecenia możemy sprawdzić za pomocą wspomnianego już wcześniej polecenia \texttt{ls} (dodatkowo użyliśmy tutaj polecenia \texttt{grep}, opisanego na stronie \pageref{sec:grep}, do wyświetlenia tylko interesującego nas wpisu).
\begin{verbatim}
[beton@xps ~]$ ls -l | grep koza
-rwx------.  1 beton beton      0 04-24 18:09 koza
\end{verbatim}
Jeżeli chcielibyśmy po prostu dodać lub usunąć konkretne prawo to z pomocą przychodzi zapis znakowy uprawnień. Symbole jakie możemy przekazywać do polecenia zamieszczono w tabelach poniżej.
\begin{table}[h!t!b]
\begin{center}
\begin{tabular}{|c|c|} \hline
Symbol & Opis \\ \hline
\texttt{u} & uprawnienia właściciela \\ \hline
\texttt{g} & uprawnienia grupy \\ \hline
\texttt{o} & uprawnienia innych \\ \hline
\texttt{a} & uprawnienia wszystkich \\ \hline
\end{tabular}
\caption{Symbole właścicieli praw akceptowane przez \texttt{chmod}}
\label{tab:chmod_symbole_wlasciciele}
\end{center}
\end{table}

\begin{table}[h!t!b]
\begin{center}
\begin{tabular}{|c|c|} \hline
Symbol & Opis \\ \hline
\texttt{+} & dodanie do istniejących uprawnień \\ \hline
\texttt{-} & odebranie z istniejących uprawnień \\ \hline
\texttt{=} & zamiana istniejących uprawnień \\ \hline
\end{tabular}
\caption{Symbole operatorów akceptowane przez \texttt{chmod}}
\label{tab:chmod_symbole_operatory}
\end{center}
\end{table}

\FloatBarrier

Uprawnienia określamy za pomocą znaków, tak jak jest to przedstawione w tabeli na stronie \pageref{tab:uprawnienia}. Dla lepszego zrozumienia posłużymy się przykładem - dodamy teraz uprawnienia czytania pliku \textit{koza} pozostałym użytkownikom.
\begin{verbatim}
[beton@xps ~]$ chmod o+r koza 
[beton@xps ~]$ ls -l | grep koza
-rwx-----x.  1 beton beton      0 04-24 18:09 koza
\end{verbatim}
Przeanalizujmy szybko powyższe polecenie - wywołaliśmy \texttt{chmod} wskazując że zmieniamy uprawnienia pozostałych użytkowników (\texttt{o}), dodając do istniejących (\texttt{+}) prawo odczytu (\texttt{r}). Niewątpliwie zapis znakowy jest bardziej czytelny niż zapis numeryczny, tym niemniej zalecamy zapamiętanie obu form, gdyż każda z nich przyda się Ci w przyszłości.

\subsection{chown}
Możliwe że zastanawiasz się teraz co zrobić w sytuacji gdy chciałbyś zmienić grupę której przypisujesz uprawnienia do pliku, lub też chciałbyś zmienić właściciela pliku. Tę funkcjonalność dostarcza nam polecenie \texttt{chown} (\textit{ang.} \textbf{ch}ange file \textbf{own}er and group).

\section{Poruszanie się w systemie plików}
\subsection{pwd}
Jak już wspomniałem większość terminali jest skonfigurowana w taki sposób aby przed znakiem zachęty wyświetlać nazwę katalogu w którym obecnie pracujemy. Jeżeli okazałoby się jednak, że terminal nie wyświetla tej informacji, lub nie jest ona dla nas wystarczająca, z pomocą przychodzi polecenie \texttt{pwd} (\textit{ang.} \textbf{p}rint name of current/\textbf{w}orking \textbf{d}irectory). Dla zademonstrowania polecenia wywołamy \texttt{pwd} będąc w katalogu domowym użytkownika \textit{beton}.
\begin{verbatim}
[beton@xps ~]$ pwd
/home/beton
\end{verbatim} 
Jak widać na przykładzie, po zastosowaniu polecenia, dowiadujemy się, że pracujemy w katalogu domowym użytkownika \textit{beton} - potwierdza to także znak \textasciitilde przed znakiem zachęty \$.

\subsection{cd}
Jedną z najczęściej wykonywanych w terminalu akcji będzie prawdopodobnie zmiana katalogu w którym obecnie pracuje użytkownik. Do tego celu służy prosta komenda \texttt{cd} (\textit{ang.} \textbf{c}hange \textbf{d}irectory). W przykładzie podanym poniżej zademonstrujemy jak przenieść się z obecnego katalogu, do katalogu \textit{/} (katalogu głównego):
\begin{verbatim}
[beton@xps ~]$ cd /
[beton@xps /]$ pwd
/
[beton@xps /]$ 
\end{verbatim}
Możemy także przenieść się do katalogu, który znajduje się w miejscu w którym jesteśmy obecnie - załóżmy, że jesteśmy w katalogu domowym, a chcemy się przenieść do katalogu \textit{Dokumenty}:
\begin{verbatim}
[beton@xps ~]$ cd Dokumenty/
[beton@xps Dokumenty]$ pwd
/home/beton/Dokumenty
[beton@xps Dokumenty]$
\end{verbatim}

\subsubsection{Etykiety \texttt{.} i \texttt{..}}
Czasem może się zdarzyć, że będziemy chcieli wskazać na katalog w którym obecnie się znajdujemy (zauważymy to zwłaszcza podczas uruchamiania plików wykonywalnych). Taką etykietą jest znak \texttt{.} (znak kropki) - możemy ją zastosować np. w połączeniu z poleceniem \texttt{cd} (posłużymy się tutaj tym samym katalogiem, co w poprzednim przypadku):
\begin{verbatim}
[beton@xps ~]$ pwd
/home/beton
[beton@xps ~]$ cd ./Dokumenty/
[beton@xps Dokumenty]$ pwd
/home/beton/Dokumenty
[beton@xps Dokumenty]$ 
\end{verbatim}
Inną etykietą o podobnym działaniu jest \texttt{..} - wskazuje ona na katalog znajdujący się wyżej w hierarchii wobec katalogu w którym obecnie się znajdujemy. Aby łatwiej zrozumieć sens istnienia takiej etykiety zademonstrujemy dwa przykłady - bez i z zastosowaniem \texttt{..}:
\begin{verbatim}
[beton@xps ~] pwd
/home/beton
[beton@xps ~] cd /home
[beton@xps ~] pwd
/home
\end{verbatim}
\begin{verbatim}
[beton@xps ~] pwd
/home/beton
[beton@xps ~] cd ..
[beton@xps ~] pwd
/home
\end{verbatim}
Przykład pierwszy pokazuje przejście do katalogu wyżej poprzez podanie bezwzględnej ścieżki do tegoż katalogu. Chociaż w przykładzie nie jest to skomplikowana ścieżka, dla innych katalogów może być to seria katalogów w katalogach; przepisywanie całości może okazać się męczące i nieefektywne. Dodatkowo podejście takie utrudnia pisanie skryptów.
\newline
\newline
Przykład drugi używa jako ścieżkę do katalogu wyżej etykiety \texttt{..} - dzięki niej nie musimy znać pełnej ścieżki do katalogu. Dodatkowo znacznie łatwiej zastosować ją w naszych skryptach.

\subsection{ls}
\label{sec:ls}
Przydatnym poleceniem do wyświetlania zawartości katalogu jest polecenie \texttt{ls} (\textit{ang.} \textbf{l}i\textbf{s}t directory contents). W przypadku wywołania \texttt{ls} bez żadnego dodatkowego argumentu ani przełącznika będzie skutkowało wyświetleniem zawartości obecnego katalogu w formie, której przykład można zobaczyć poniżej:
\begin{verbatim}
[beton@xps ~]$ ls
Dokumenty  Obrazy   Programy   Pulpit    Wideo
Muzyka     Pobrane  Publiczny  Szablony
\end{verbatim}
W tym przypadku polecenie wywołane było w katalogu domowym użytkownika \textit{beton} (nie przejmujcie się, jeżeli wasz katalog domowy wygląda inaczej). \newline

Komenda posiada wiele przełączników, jednym z nich jest używany bardzo często \textbf{-l} powodujący wyświetlanie każdego elementu w katalogu wraz ze szczegółami - m.in prawami dostępu, właścicielem i datą ostatniej modyfikacji.
\begin{verbatim}
[beton@xps ~]$ ls -l
razem 44
drwxr-xr-x.  9 beton beton 4096 04-07 13:27 Dokumenty
drwxr-xr-x. 11 beton beton 4096 04-06 11:00 Muzyka
drwxr-xr-x.  6 beton beton 4096 04-03 17:34 Obrazy
drwxr-xr-x.  9 beton beton 4096 04-10 16:26 Pobrane
drwxrwxr-x.  5 beton beton 4096 03-12 10:19 Programy
drwxr-xr-x.  2 beton beton 4096 03-01 20:41 Publiczny
drwxr-xr-x.  2 beton beton 4096 04-10 08:37 Pulpit
drwxr-xr-x.  2 beton beton 4096 03-01 20:41 Szablony
drwxr-xr-x.  6 beton beton 4096 03-13 12:17 Wideo
\end{verbatim}
Kolejnym przełącznikiem jaki wyróżnimy poprzez pokazanie Ci jego działania, jest przełącznik \textbf{-a}, wyłączający wyświetlanie plików i katalogów ukrytych (w systemach z rodziny Linuks takie pliki i katalogi zaczynają swoje nazwy od znaku kropki.
\begin{verbatim}
[beton@xps ~]$ ls -a
.                    .grl-podcasts              Pulpit
..                   .gstreamer-0.10            .pulse
.abrt                .gtk-bookmarks             .pulse-cookie
.adobe               .gvfs                      .setroubleshoot
.android             .htoprc                    .shotwell
.bash_history        .ICEauthority              .Skype
.bash_logout         .icons                     .ssh
.bash_profile        .imsettings.log            streching.png
\end{verbatim}
\subsubsection*{Przykładowe przełączniki}
\begin{description}
\item \textbf{-a} lub \textbf{--all} wyłącza ukrywania katalogów zaczynających się od .
\item \textbf{-l} użycie długiego formatu wyjściowego wyświetlającego wynik w formie listy
\end{description}

\subsection{uruchamianie skryptów i plików wykonywalnych}
Uruchamianie programów może się odbywać na kilka sposobów, z jednego z nich korzystaliśmy nawet kilka razy podczas opisywania poleceń powyżej. Jeżeli katalog w którym znajduje się nasz program jest dopisany do zmiennej środowiskowej \texttt{PATH} (co to jest zmienna środowiskowa wyjaśnimy Ci później), to wystarczy podać nazwę pliku, a program się wykona. Dokładnie w taki sposób wywoływaliśmy polecenie \texttt{pwd}. Dla zademonstrowania działania zmiennej środowiskowej wyświetlę Ci jej przykładową zawartość (o wyświetlaniu tekstu na ekranie konsoli dowiesz się w następnych rozdziałach).
\begin{verbatim}
[beton@xps ~]$ echo $PATH
/usr/local/bin:/usr/bin:/bin
\end{verbatim}
W powyższym przypadku zmienna wskazuje na trzy katalogi: \texttt{/usr/local/bin}, \texttt{/usr/bin} i \texttt{/bin}. Oznacza to, że do wszystkich programów umieszczonych wewnątrz nich można się odwołać poprzez podanie nazwy pliku z programem.
\newline
Spróbujmy teraz innego podejścia - programy można uruchamiać również poprzez podanie adresu bezwzględnego\index{adres bezwzględny} do pliku wykonywalnego (oznacza to podanie pełnej ścieżki do pliku, z uwzględnieniem katalogu głównego). Dla przykładu wywołamy tak polecenie \texttt{pwd} (znajduje się ono w \texttt{/bin}).
\begin{verbatim}
[beton@xps ~]$ /bin/pwd
/home/beton
\end{verbatim}
Ostatnim podejściem które będziemy stosować jest podawanie adresu względnego\index{adres względny} do pliku (oznacza to, że podawać będziemy ścieżkę do pliku względem katalogu w którym pracujemy obecnie). Wywołanie jest bardzo podobne do używania adresu bezwzględnego, jednak ścieżkę do pliku poprzedzamy znakiem kropki. Tak jak w poprzednich dwóch przykładach posłużymy się poleceniem \texttt{pwd} - różnica będzie taka, że program będziemy wywoływać z katalogu w którym znajduje się plik wykonywalny.
\begin{verbatim}
[beton@xps ~]$ cd /bin
[beton@xps bin]$ ./pwd 
/bin
[beton@xps bin]$ 
\end{verbatim}
Istotną rzeczą o której należy pamiętać przy wykonywaniu plików jest posiadanie uprawnień do wykonywania danego programu - jeżeli nie będziemy ich posiadać, to wywołanie zakończy się błędem (o uprawnieniach możesz przeczytać w rozdziale \ref{sec:uprawnienia}).
\newline
\newline
Wywoływanie skryptów Basha odbywa się podobnie - możemy próbować uruchomić go tak jak plik wykonywalny, jednak powiedzie się to tylko jeśli będziemy posiadali do tego prawa. Jeśli jednak okaże się, że tak nie jest, możesz spróbować innego wywołania. Podając przed nazwą skryptu, nazwę interpretera który ma wykonać nasz kod możemy uruchomić program bez większych problemów. Przykładowo interpreterem z którego będziemy korzystać będzie interpreter \texttt{bash}. Załóżmy, że posiadamy skrypt o nazwie \textit{Hello.sh} (program będzie jedynie wyświetlał na ekranie tekst 
"\textit{Hello world!}") i chcemy go uruchomić w poznany przed chwilą sposób.
\begin{verbatim}
[beton@xps tmp]$ bash Hello.sh 
Hello world!
\end{verbatim} 
Wsród uprawnień jakie były przypisane do pliku nie było praw wykonywania, co możesz zobaczyć na poniższym listingu.
\begin{verbatim}
-rw-rw-r--. 1 beton beton     33 04-12 23:46 Hello.sh
\end{verbatim}
Jaki zatem nasuwa się wniosek? Jeżeli chcemy uruchomić plik wykonywalny, to musimy posiadać prawa wykonywania. Przewagą skryptu jest to, że wykonywalny musi być jedynie interpreter, do skryptu musimy mieć jedynie prawa odczytu. Przydatne, nieprawdaż?

\section{Wybrane operacje na plikach i katalogach}
W poniższej sekcji zawarliśmy wybrane polecenia służące do operacji na plikach i katalogach. Wraz z krótkim opisem i przykładowym wywołaniem podaliśmy przełączniki, które uznaliśmy za przydatne podczas nauki. Jeżeli chcesz wiedzieć o poleceniu więcej, skorzystaj z dostarczonej wraz z systemem pomocy.

\subsection{touch}
Jednym ze sposobów na utworzenie pustego pliku z poziomu wiersza poleceń jest użycie komendy \texttt{touch} (jest to właściwie efekt uboczny działania tego polecenia, ale mimo to bardzo przydatny). Przykład utworzenia pliku \textit{koza} będzie wyglądał następująco:
\begin{verbatim}
[beton@xps ~]$ ls
Dokumenty  Obrazy   Programy   Pulpit    Wideo
Muzyka     Pobrane  Publiczny  Szablony
[beton@xps ~]$ touch koza
[beton@xps ~]$ ls
Dokumenty  Muzyka  Pobrane   Publiczny  Szablony  koza
Obrazy  Programy   Pulpit    Wideo
\end{verbatim}
Jeżeli chcemy utworzyć na raz kilka plików możemy je utworzyć jednym poleceniem, oddzielając spacjami kolejne nazwy:
\begin{verbatim}
[beton@xps ~]$ touch plik1 plik2 plik3
\end{verbatim}
\subsubsection*{Przykładowe przełączniki}
%TODO Add more info about modyfing timestamps

\subsection{mv}
Aby móc przenieść pliki i katalogi, lub zmienić nazwę danego obiektu skorzystamy z polecenia \texttt{mv} (\textit{ang.} \textbf{m}o\textbf{v}e). Polecenie wywołujemy z co najmniej dwoma parametrami - pierwszym jest plik źródłowy, a drugim jest plik lub katalog do którego ma trafić plik źródłowy. Przykładowo aby zmienić nazwę pliku z \textit{koza} na \textit{kózka} wywołamy polecenie \texttt{mv} w następujący sposób:
\begin{verbatim}
[beton@xps ~]$ mv koza kózka
\end{verbatim}
Podobnie wygląda przenoszenie plików - aby przenieść plik \textit{koza} znajdujący się w katalogu \textit{łąka} do katalogu \textit{zagroda} wywołamy \texttt{mv} w następujący sposób:
\begin{verbatim}
[beton@xps ~]$ mv łąka/koza zagroda/
\end{verbatim}
Jak już wcześniej wspomniałem, \texttt{mv} jest wywoływany z co najmniej dwoma parametrami - oznacza to, że możemy określić wiele plików lub katalogów które chcemy przenieść (miejsce docelowe może być jednak tylko jedno).
\begin{verbatim}
[beton@xps ~]$ mv plik1 plik2 plik3 katalog_na_pliki/
\end{verbatim}
\subsubsection*{Przykładowe przełączniki}
\begin{description}
\item \textbf{-u} lub \textbf{--update} powoduje przenoszenie tylko gdy przenoszony plik jest nowszy niż plik docelowy lub gdy plik docelowy nie istnieje.
\end{description}

\subsection{cp}
Jeżeli zechcemy skopiować pliki lub katalogi, z pomocą przyjdzie nam polecenie \texttt{cp} (\textit{ang.} \textbf{c}o\textbf{p}y files and directories). Obsługa polecenia jest bardzo podobna do obsługi \texttt{mv} - identyczne jest wywołanie, w którym podajemy najpierw nazwę polecenia, następnie plik źródłowy i plik do którego ma zostać on skopiowany. Możliwe jest także skopiowanie kilku plików naraz, jednak wtedy zamiast pliku docelowego należy podać katalog, w którym zostaną umieszczone kopie. Poniższy kod zademonstruje sytuację w której skopiujemy 4 pliki do jednego katalogu.
\begin{verbatim}
[beton@xps tmp]$ cp p1 p2 p3 p4 na_pliki
[beton@xps tmp]$ ls na_pliki/
p1  p2  p3  p4
[beton@xps tmp]$ 
\end{verbatim}
Co by się stało gdybyśmy spróbowali skopiować te same pliki do innego pliku?
\begin{verbatim}
[beton@xps tmp]$ cp p1 p2 p3 p4 p5
cp: cel `p5' nie jest katalogiem
[beton@xps tmp]$ 
\end{verbatim}
\subsubsection*{Przykładowe przełączniki}
%TODO add switches

\subsection{rm}
Do skasowania plików używamy polecenia \texttt{rm}(\textit{ang.} \textbf{r}e\textbf{m}ove)
\subsubsection*{Przykładowe przełączniki}
\begin{description}
\item \textbf{-r} lub \textbf{-R} lub \textbf{--recursive} usuwanie rekursywne, działa także na katalogi posiadające zawartość.
\end{description}

\subsection{mkdir}
Odpowiednikiem dla polecenia touch do tworzenia jest polecenie \texttt{mkdir} (\textit{ang.} \textbf{m}a\textbf{k}e \textbf{dir}ectories. Aby utworzyć przykładowy katalog \textit{koza} za pomocą wiersza poleceń skorzystamy z komendy:
\begin{verbatim}
[beton@xps ~]$ mkdir koza
\end{verbatim}
Oczywiście, jeżeli chcielibyśmy utworzyć taki katalog wewnątrz innego katalogu, np. \textit{łąka}, jako parametr przekażemy ścieżkę do tego katalogu:
\begin{verbatim}
[beton@xps ~]$ mv łąka/koza
\end{verbatim}
Jeżeli jednak próbowalibyśmy utworzyć katalog w miejscu które nie istnieje, np. łąkaa wydając polecenie:
\begin{verbatim}
[beton@xps ~]$ mv łąkaa/koza
\end{verbatim}
to bez zastosowania dodatkowego przełącznika \textbf{-p} konsola zwróci błąd z komunikatem:
\begin{verbatim}
mkdir: nie można utworzyć katalogu `łąkaa/koza': Nie ma takiego pliku ani katalogu
\end{verbatim}
\subsubsection*{Przykładowe przełączniki}
%TODO add switches

\subsection{rmdir}
Jak łatwo się domyślić - jeżeli istnieje polecenie \texttt{mkdir}, służące do tworzenia katalogów, to istnieje również polecenie \texttt{rmdir} (\textit{ang.} \textbf{r}e\textbf{m}ove empty \textbf{dir}ectories), służące do kasowania pustych katalogów. Wywołanie operacji usuwania katalogu \textit{koza} będzie identyczne jak podczas tworzenia go:
\begin{verbatim}
[beton@xps ~]$ rmdir koza
\end{verbatim}
Pozostałe operacje jak kasowanie katalogu wewnątrz innego katalogu oraz usuwanie całej hierarchii katalogów jest analogiczne do operowania poleceniem \texttt{mkdir}.
\newline
\subsubsection*{Przykładowe przełączniki}
\begin{description}
\item \textbf{-p} lub \textbf{--parents} usuwa katalog wraz z jego katalogami nadrzędnymi
%TODO add more switches
\end{description}

\section{Edytory tekstu}
\subsection{nano i pico}
Do prostej edycji plików z poziomu wiersza poleceń możesz skorzystać z programu \texttt{nano} (który jest rozbudowanym klonem edytora \texttt{pico}).
\subsection{Vi i Vim}
Edytorami które oferując znacznie więcej możliwości (ale są zarazem znacznie bardziej skomplikowane) są \texttt{vi} i \texttt{vim} (\textit{ang.} \textbf{V}i \textbf{IM}proved).