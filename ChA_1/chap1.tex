\documentclass[10 pt, a4paper, draft]{report}
\usepackage{polski}
\usepackage[utf8]{inputenc}
\usepackage{amsmath}
\usepackage{amssymb}
\usepackage{amsfonts}
\usepackage{draftwatermark}
\usepackage{makeidx}
\usepackage{hyperref}

\author{Bartłomiej Sacharski}
\title{WIZUT Bash Guide - Rozdział I}
\makeindex

\begin{document}
\maketitle
\tableofcontents
\newpage

\chapter{Podstawy operowania wierszem poleceń}
Drogi Czytelniku! \newline
Witamy w pierwszym rozdziale krótkiego skryptu . %TODO dodać tytuł
W tej sekcji spróbujemy zapoznać Cię z podstawami obsługi wiersza poleceń systemów z rodziny Linuks. Oczywiście, rozumiemy Twoje zdziwienie - jak w czasach monitorów Full HD i interfejsów dotykowych, można korzystać jeszcze z czegoś tak przestarzałego jak wiersz poleceń. Nic bardziej mylnego - tryb tekstowy ma się bardzo dobrze i jest używany przez miliony osób na całym świecie. Gwarantujemy że po zapoznaniu się z całą książką sam stwierdzisz że niektóre operacje szybciej jest "wystukać" niż "wyklikać".

\section{Pierwsze spotkanie z konsolą}
Podczas swojej nauki będziesz korzystał z tzw. emulatora terminala\index{terminal} - w dawnych czasach było to urządzenie pozwalające na pracę z komputerem, często była to klawiatura służąca do wprowadzania informacji oraz monitor lub drukarka do wyprowadzania danych. Wraz z postępem technologicznym terminale zostały zastąpione wirtualnymi odpowiednikami - właśnie takim wirtualnym terminalem będziemy się posługiwać. Bardzo możliwe że będziesz korzystać z PuTTY, Konsole lub gnome-terminal. Jeśli jednak Twój program nazywa się inaczej to nie przejmuj się - nie ma to w tym przypadku większego znaczenia.

\subsection{Tuż po uruchomieniu}
Ze sporą dozą prawdopodobieństwa można założyć, że na ekranie zostanie wyświetlony tekst wyglądający mniej-więcej tak:
\begin{verbatim}
[beton@xps ~]$ 
\end{verbatim}
W tym krótkim tekście zawartych jest sporo informacji - pracujemy obecnie jako użytkownik \textit{beton} na maszynie zwanej \textit{xps} i katalogiem w którym obecnie się znajdujemy jest katalog domowy (oznaczany często znakiem \textasciitilde). Kolejnym znakiem który informuje nas o stanie terminala jest znak \$ (\index{znak zachęty}nazywany znakiem zachęty). Pojawienie się tego znaku oznacza że terminal nie wykonuje w tym momencie żadnej operacji i można wprowadzić nowe polecenie. Co ciekawe, jeżeli zdarzy się Wam pracować na koncie z uprawnieniami administratora systemu \index{root}(konto root) zauważycie pewnie że zamiast znaku zachęty \$ pojawi się znak \# - takie rozróżnienie jest często wykorzystywane w instrukcjach opisujących wykonanie czegoś za pomocą wiersza poleceń; polecenia wykonywane za pomocą konta administratora będą się zaczynały właśnie od tego znaku.

\section{Poruszanie się w systemie plików}
\subsection{pwd}
Jak już wspomniałem większość terminali jest skonfigurowana w taki sposób aby przed znakiem zachęty wyświetlać nazwę katalogu w którym obecnie pracujemy. Jeżeli okazałoby się jednak, że terminal nie wyświetla tej informacji, lub nie jest ona dla nas wystarczająca, z pomocą przychodzi polecenie \texttt{pwd} (\textbf{p}rint name of current/\textbf{w}orking \textbf{d}irectory). Dla zademonstrowania polecenia wywołamy \texttt{pwd} będąc w katalogu domowym użytkownika \textit{beton}.
\begin{verbatim}
[beton@xps ~]$ pwd
/home/beton
\end{verbatim} 
Jak widać na przykładzie, po zastosowaniu polecenia, dowiadujemy się że pracujemy w katalogu domowym użytkownika \textit{beton} - potwierdza to także znak \textasciitilde przed znakiem zachęty \$.

\subsection{cd}
Jedną z najczęściej wykonywanych w terminalu akcji będzie prawdopodobnie zmiana katalogu w którym obecnie pracuje użytkownik. Do tego celu służy prosta komenda \texttt{cd} (\textbf{c}hange \textbf{d}irectory). W przykładzie podanym poniżej zademonstrujemy jak przenieść się z obecnego katalogu, do katalogu \textit{/} (katalogu głównego):
\begin{verbatim}
[beton@xps ~]$ cd /
[beton@xps /]$ pwd
/
[beton@xps /]$ 
\end{verbatim}
%TODO possibly add cd switches

\subsection{ls}
Przydatnym poleceniem do wyświetlania zawartości katalogu jest polecenie \texttt{ls} (\textbf{l}i\textbf{s}t directory contents). W przypadku wywołania \texttt{ls} bez żadnego dodatkowego argumentu ani przełącznika będzie skutkowało wyświetleniem zawartości obecnego katalogu w prostej formie, której przykład można zobaczyć poniżej:
\begin{verbatim}
[beton@xps ~]$ ls
Dokumenty  Obrazy   Programy   Pulpit    Wideo
Muzyka     Pobrane  Publiczny  Szablony
\end{verbatim}
W tym przypadku polecenie wywołane było w katalogu domowym użytkownika \textit{beton} (nie przejmujcie się, jeżeli wasz katalog domowy wygląda inaczej). \newline

Komenda posiada wiele przełączników, jednym z nich jest używany bardzo często \textbf{-l} powodujący wyświetlanie każdego elementu w katalogu wraz ze szczegółami - m.in prawami dostępu, właścicielem i datą ostatniej modyfikacji.
\begin{verbatim}
[beton@xps ~]$ ls -l
razem 44
drwxr-xr-x.  9 beton beton 4096 04-07 13:27 Dokumenty
drwxr-xr-x. 11 beton beton 4096 04-06 11:00 Muzyka
drwxr-xr-x.  6 beton beton 4096 04-03 17:34 Obrazy
drwxr-xr-x.  9 beton beton 4096 04-10 16:26 Pobrane
drwxrwxr-x.  5 beton beton 4096 03-12 10:19 Programy
drwxr-xr-x.  2 beton beton 4096 03-01 20:41 Publiczny
drwxr-xr-x.  2 beton beton 4096 04-10 08:37 Pulpit
drwxr-xr-x.  2 beton beton 4096 03-01 20:41 Szablony
drwxr-xr-x.  6 beton beton 4096 03-13 12:17 Wideo
\end{verbatim}
Kolejnym przełącznikiem jaki wyróżnimy poprzez pokazanie Ci jego działania, jest przełącznik \textbf{-a}, wyłączający wyświetlanie plików i katalogów ukrytych (w systemach z rodziny Linuks takie pliki i katalogi zaczynają swoje nazwy od znaku kropki.
\begin{verbatim}
[beton@xps ~]$ ls -a
.                    .grl-podcasts              Pulpit
..                   .gstreamer-0.10            .pulse
.abrt                .gtk-bookmarks             .pulse-cookie
.adobe               .gvfs                      .setroubleshoot
.android             .htoprc                    .shotwell
.bash_history        .ICEauthority              .Skype
.bash_logout         .icons                     .ssh
.bash_profile        .imsettings.log            streching.png
\end{verbatim}
\subsubsection*{Przełączniki}
\begin{description}
\item \textbf{-a} lub \textbf{--all} wyłącza ukrywania katalogów zaczynających się od .
\item \textbf{-l} użycie długiego formatu wyjściowego wyświetlającego wynik w formie listy
\end{description}

\subsection{. i ..}
Wprawne oko zauważy pewnie dwa elementy w poprzednim listingu które nie były dotychczas wytłumaczone - \texttt{.} i \texttt{..} - czasem zdarzy się że będziemy chcieli wskazać na katalog w którym obecnie się znajdujemy (zauważymy to zwłaszcza podczas uruchamiania plików wykonywalnych). Taką etykietą jest znak \texttt{.} (znak kropki).
%TODO add more about .
\newline
\newline
Etykieta \texttt{..} wskazuje na katalog znajdujący się wyżej w hierarchii wobec katalogu w którym obecnie się znajdujemy. Aby łatwiej zrozumieć sens istnienia takiej etykiety zademonstrujemy dwa przykłady - bez i z zastosowaniem \texttt{..}:
\newline
\begin{verbatim}
[beton@xps ~] pwd
/home/beton
[beton@xps ~] cd /home
[beton@xps ~] pwd
/home
\end{verbatim}
\begin{verbatim}
[beton@xps ~] pwd
/home/beton
[beton@xps ~] cd ..
[beton@xps ~] pwd
/home
\end{verbatim}
Przykład pierwszy pokazuje przejście do katalogu wyżej poprzez podanie bezwzględnej ścieżki do tegoż katalogu. Chociaż w przykładzie nie jest to skomplikowana ścieżka, dla innych katalogów może być to seria katalogów w katalogach; przepisywanie całości może okazać się męczące i nieefektywne. Dodatkowo podejście takie utrudnia pisanie skryptów.
\newline
\newline
Przykład drugi używa jako ścieżkę do katalogu wyżej etykiety \texttt{..} - dzięki niej nie musimy znać pełnej ścieżki do katalogu. Dodatkowo znacznie łatwiej zastosować ją w naszych skryptach.

\subsection{uruchamianie programów i plików wykonywalnych}
Uruchamianie programów może się odbywać na kilka sposobów, z jednego z nich korzystaliśmy nawet kilka razy podczas opisywania poleceń powyżej. Jeżeli katalog w którym znajduje się nasz program jest dopisany do zmiennej środowiskowej \texttt{PATH} (co to jest zmienna środowiskowa wyjaśnimy Ci później), to wystarczy podać nazwę pliku, a program się wykona. Dokładnie w taki sposób wywoływaliśmy polecenie \texttt{pwd}. Dla zademonstrowania działania zmiennej środowiskowej wyświetlę Ci jej przykładową zawartość (o wyświetlaniu tekstu na ekranie konsoli dowiesz się w następnych rozdziałach).
\begin{verbatim}
[beton@xps ~]$ echo $PATH
/usr/local/bin:/usr/bin:/bin
\end{verbatim}
W powyższym przypadku zmienna wskazuje na trzy katalogi: \texttt{/usr/local/bin}, \texttt{/usr/bin} i \texttt{/bin}. Oznacza to że do wszystkich programów umieszczonych wewnątrz nich można się odwołać poprzez podanie nazwy pliku z programem.
\newline
Spróbujmy teraz innego podejścia - programy można uruchamiać również poprzez podanie adresu bezwzględnego\index{adres bezwzględny} do pliku wykonywalnego (oznacza to podanie pełnej ścieżki do pliku, z uwzględnieniem katalogu głównego). Dla przykładu wywołamy tak polecenie \texttt{pwd} (znajduje się ono w \texttt{/bin}).
\begin{verbatim}
[beton@xps ~]$ /bin/pwd
/home/beton
\end{verbatim}
Ostatnim podejściem które będziemy stosować jest podawanie adresu względnego\index{adres względny} do pliku (oznacza to że podawać będziemy ścieżkę do pliku względem katalogu w którym pracujemy obecnie). Wywołanie jest bardzo podobne do używania adresu bezwzględnego, jednak ścieżkę do pliku poprzedzamy znakiem kropki. Tak jak w poprzednich dwóch przykładach posłużymy się poleceniem \texttt{pwd} - różnica będzie taka, że program będziemy wywoływać z katalogu w którym znajduje się plik wykonywalny.
\begin{verbatim}
[beton@xps ~]$ cd /bin
[beton@xps bin]$ ./pwd 
/bin
[beton@xps bin]$ 
\end{verbatim}
Istotną rzeczą o której należy pamiętać przy wykonywaniu plików jest posiadanie uprawnień do wykonywania danego programu - jeżeli nie będziemy ich posiadać, to wywołanie zakończy się błędem (o uprawnieniach możesz przeczytać w rozdziale ...) %TODO add reference to the section containing file rights.

\section{Proste operacje na plikach i katalogach}
\subsection{touch}
Jednym ze sposobów na utworzenie pustego pliku z poziomu wiersza poleceń jest użycie komendy \texttt{touch} (jest to właściwie efekt uboczny działania tego polecenia, ale mimo to bardzo przydatny). Przykład utworzenia pliku \textit{koza} będzie wyglądał następująco:
\begin{verbatim}
[beton@xps ~]$ ls
Dokumenty  Obrazy   Programy   Pulpit    Wideo
Muzyka     Pobrane  Publiczny  Szablony
[beton@xps ~]$ touch koza
[beton@xps ~]$ ls
Dokumenty  Muzyka  Pobrane   Publiczny  Szablony  koza
Obrazy  Programy   Pulpit    Wideo
\end{verbatim}
Jeżeli chcemy utworzyć na raz kilka plików możemy je utworzyć jednym poleceniem, oddzielając spacjami kolejne nazwy:
\begin{verbatim}
[beton@xps ~]$ touch plik1 plik2 plik3
\end{verbatim}
\subsubsection*{Przełączniki}
%TODO Add more info about modyfing timestamps

\subsection{mv}
Aby móc przenieść pliki i katalogi, lub zmienić nazwę danego obiektu skorzystamy z polecenia \texttt{mv} (\textbf{m}o\textbf{v}e). Polecenie wywołujemy z co najmniej dwoma parametrami - pierwszym jest plik źródłowy, a drugim jest plik lub katalog do którego ma trafić plik źródłowy. Przykładowo aby zmienić nazwę pliku z \textit{koza} na \textit{kózka} wywołamy polecenie \texttt{mv} w następujący sposób:
\begin{verbatim}
[beton@xps ~]$ mv koza kózka
\end{verbatim}
Podobnie wygląda przenoszenie plików - aby przenieść plik \textit{koza} znajdujący się w katalogu \textit{łąka} do katalogu \textit{zagroda} wywołamy \texttt{mv} w następujący sposób:
\begin{verbatim}
[beton@xps ~]$ mv łąka/koza zagroda/
\end{verbatim}
Jak już wcześniej wspomniałem, \texttt{mv} jest wywoływany z co najmniej dwoma parametrami - oznacza to że możemy określić wiele plików lub katalogów które chcemy przenieść (miejsce docelowe może być jednak tylko jedno).
\begin{verbatim}
[beton@xps ~]$ mv plik1 plik2 plik3 katalog_na_pliki/
\end{verbatim}
%TODO write about mv
\subsubsection*{Przełączniki}
%TODO add switches

\subsection{cp}
%TODO write about cp

\subsection{rm}
Do skasowania plików używamy polecenia \texttt{rm}(\textbf{r}e\textbf{m}ove)
\subsubsection*{Przełączniki}
%TODO add switches

\subsection{mkdir}
Odpowiednikiem dla polecenia touch do tworzenia jest polecenie \texttt{mkdir} (\textbf{m}a\textbf{k}e \textbf{dir}ectories. Aby utworzyć przykładowy katalog \textit{koza} za pomocą wiersza poleceń skorzystamy z komendy:
\begin{verbatim}
[beton@xps ~]$ mkdir koza
\end{verbatim}
Oczywiście, jeżeli chcielibyśmy utworzyć taki katalog wewnątrz innego katalogu, np. \textit{łąka}, jako parametr przekażemy ścieżkę do tego katalogu:
\begin{verbatim}
[beton@xps ~]$ mv łąka/koza
\end{verbatim}
Jeżeli jednak próbowalibyśmy utworzyć katalog w miejscu które nie istnieje, np. łąkaa wydając polecenie:
\begin{verbatim}
[beton@xps ~]$ mv łąkaa/koza
\end{verbatim}
to bez zastosowania dodatkowego przełącznika \textbf{-p} konsola zwróci błąd z komunikatem:
\begin{verbatim}
mkdir: nie można utworzyć katalogu `łąkaa/koza': Nie ma takiego pliku ani katalogu
\end{verbatim}
\subsubsection*{Przełączniki}
%TODO add switches

\subsection{rmdir}
Jak łatwo się domyślić - jeżeli istnieje polecenie \texttt{mkdir}, służące do tworzenia katalogów, to istnieje również polecenie \texttt{rmdir} (\textbf{r}e\textbf{m}ove empty \textbf{dir}ectories), służące do kasowania pustych katalogów. Wywołanie operacji usuwania katalogu \textit{koza} będzie identyczne jak podczas tworzenia go:
\begin{verbatim}
[beton@xps ~]$ rmdir koza
\end{verbatim}
Pozostałe operacje jak kasowanie katalogu wewnątrz innego katalogu oraz usuwanie całej hierarchii katalogów jest analogiczne do operowania poleceniem \texttt{mkdir}.
\newline
\subsubsection*{Przełączniki}
\begin{description}
\item \textbf{-p} lub \textbf{--parents} usuwa katalog wraz z jego katalogami nadrzędnymi
%TODO add more switches
\end{description}

\section{Edytory tekstu}
\subsection{nano i pico}
Do prostej edycji plików z poziomu wiersza poleceń możesz skorzystać z programu \texttt{nano} (który jest rozbudowanym klonem edytora \texttt{pico}).
\subsection{Vi i Vim}
Edytorami które oferując znacznie więcej możliwości (ale są zarazem znacznie bardziej skomplikowane) są \texttt{vi} i \texttt{vim} (\textbf{V}i \textbf{IM}proved).

\section{Uzyskiwanie pomocy}
\subsection{man}
W przypadku gdy chcemy dowiedzieć się jak działa polecenie, lub sprawdzić sposób jego wywołania możemy spróbować wywołać polecenie z przełącznikiem \textbf{-?} lub \textbf{--help}. Wiele programów wyświetla wtedy bardzo ogólną informację o dostępnych przełącznikach, ich działaniu i sposobie wywołania programu. Czasem jednak potrzebujemy bardziej rozbudowanej dokumentacji - wtedy warto spróbować skorzystać z polecenia \texttt{man}(od angielskiego słowa \textbf{man}ual).
\subsubsection*{Przełączniki}
%TODO add switches
\subsection{info}
\subsection{apropos}
W przypadku gdy chcemy coś zrobić, ale nie wiemy jakiego polecenia użyć, możemy skorzystać z polecenia \texttt{apropos}. Jako argument wywołania podajemy wówczas frazę jakiej poszukujemy w instrukcji, a \texttt{apropos} przeszuka wszystkie dostępne na dysku instrukcje pod kątem tejże frazy. Przykładowo, jeżeli chcielibyśmy dowiedzieć się jakie polecenie odpowiada za tworzenie katalogu  użyjemy nowo poznanej komendy z argumentem \textit{tworzy katalog}:
\newline
%TODO add this

\clearpage
\addcontentsline{toc}{chapter}{Skorowidz}
\printindex

\end{document}
