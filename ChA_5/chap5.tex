\documentclass[10 pt, a4paper]{article}
\usepackage{polski}
\usepackage[utf8]{inputenc}
\usepackage{verbatim}
\usepackage{listings}

\usepackage{color}
\usepackage{textcomp}
\definecolor{lbcolor}{rgb}{0.9,0.9,0.9}
\lstset{
	%backgroundcolor=\color{lbcolor},
	tabsize=4,
	%rulecolor=,
    %basicstyle=\scriptsize,
    upquote=true,
    aboveskip={1.5\baselineskip},
    columns=fixed,
    showstringspaces=false,
    extendedchars=true,
    breaklines=true,
    %frame=single,
    showtabs=false,
    showspaces=false,
    showstringspaces=false,
    identifierstyle=\ttfamily,
    numbers=left, 
    numberstyle=\tiny, 
    stepnumber=1, 
    numbersep=5pt,
}
%TODO: Configure listing.

\author{Krzysztof Czajkowski}
\title{WIZUT Bash Guide - Rozdział V\\
Podstawy programowania proceduralnego}

\begin{document}
\maketitle
\tableofcontents
\newpage
\section{Wstęp}
W tym rozdziale dowiesz się, jakie korzyści płyną z używania funkcji, kiedy warto ich używać oraz ja je tworzyć.
\section{Dlaczego warto używać funkcji}
Czasami w skrypcie zdarza się, że kod wielokrotnie powtarza się. Aby uprościć jego użycie
oraz poprawić czytelność kodu można użyć funkcji, są to małe "podskrypty" działające w
ramach głównego skryptu.
\section{Tworzenie własnej funkcji}
\subsection{Składnia}
\begin{lstlisting}
function nazwa(){
	Zestaw instrukcji;
}
\end{lstlisting}
\section{Różnice w argumentach przekazywanych do funkcji a do skryptu}

\end{document}